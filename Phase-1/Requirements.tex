\chapter{Requirements}
\section{Background}
\subsection{Elicitation and Negotiation}
Requirements are fundamental to the software engineering process.
In order for us to deliver a piece of software which is as close to our customer’s vision as possible, we needed to begin the requirement engineering process, starting with elicitation.
We first analysed the brief as a team, and made a note of all the requirements which were clearly stated.
We also highlighted parts of the brief which we thought were ambiguous and needed further input from the customer to ascertain what they required more precisely.
We took these points to an interview with the customer to gain further insight into what they required.
At the interview, the customer told us that we could decide what we thought was best for some of the details, such as the exact features of the map, and the amount of customisation that can be applied to a plot.
We looked at all of these as a team and decided on an initial plan of what these should be.

We compiled all of this into the first draft of our requirements, with a section for those things which would be nice to have, but were not essential according to the brief.
This means that if we need to revisit our requirements during development because otherwise we would not be able to complete the project on time, we know which requirements can be changed or toned down to allow for this.

We took this first draft of the requirements back to the customer, and together we went through them all to ensure that we were agreed on what the requirements were.


\subsection{Presentation}
We will be using user stories, as detailed in Sommerville~\cite{sommerville2016software}, these stories will detail the needs of the user for specific features and aspects of the system.
We are using user stories as they are part of the agile development process and easily allow for us to iteratively adapt to change.
This is useful the requirements themselves change or if we realise we need to update them due to contradictions.
The user stories will also include acceptability criteria, this means that we will be easily laid out everything the system needs to be able to do.
This will also feed into the test driven development we will be using later in the project as we able to see if a requirement has been fully implemented.
We will not be using use cases as they are part of none agile development methods and more suited to transactional processes and are therefore inappropriate for this project. 

We will then tabulate our requirements to make them more accessible to us and to the customer.
In the table we will have the number and name of the user story to allow us to easily reference them.
We will then have the user story itself.
Finally, we will have the acceptance criteria split up into functional and non-functional requirements.
When referencing requirements they will be referred to by a series of 3 numbers separated by points.
The first number will represent the position of the user story associated with the requirement in the requirements table.
The next number will show if the requirement is functional, represented by a 1, or non-functional, represented by a  2.
The final number will represent the position of the requirement in it’s cell in the table.
We chose to tabulate our requirements after initially presenting our requirements to the customer as a user story followed by a list [reference to appendix] but changed them as he said it would be easier for both himself and us to read and reference.


% \section{User Stories} (in requirements-table.tex)
\newgeometry{left=0.75in, right=0.75in, top=1in, bottom=1in}
\begin{landscape}
	\section{User Stories}
	\begin{longtable}{|l||l|p{4cm}|p{8cm}|p{7cm}|}
		\hline
		\# & Title & Story & Functional & Non-functional \\ \hline \hline \endhead
		1 & GUI & As a player I must be able to see a GUI consisting of a map subdivided into plots and should be able to gain information about the state of my freehold and my individual plots.
		& \begin{enumerate}[label=1.1.\arabic*.]
			\item The entire map should be available to the user
			\item Information about individual plots should be shown:
			\begin{enumerate}
				\item Which player owns a plot
				\item Output of ore, energy, and food
				\item Robiticons installed
			\end{enumerate}
		\end{enumerate}
		& \begin{enumerate}[label=1.2.\arabic*.]
			\item The map must represent the university of York with at least 3 identifiable landmarks
			\item The map must be split into multiple evenly sized plots
			\item Each player must be uniquely identifiable on the map
			\item The player must be able to access the whole map
			\item The GUI should load in less than 5 seconds
		\end{enumerate} \\ \hline
	
	2 & Purchasing Land
	& As a player I must be able to purchase plots of land to increase the size and productivity of my freehold
	& \begin{enumerate}[label=2.1.\arabic*.]
		\item The player must be able to exchange currency for more land during the acquisition phase of the round
	\end{enumerate}
	& \begin{enumerate}[label=2.2.\arabic*.]
		\item Plots will have different strengths and weaknesses in terms of production based on location and terrain type
		\item The player must be able to cancel a purchase to avoid accidental purchases
	\end{enumerate} \\ \hline

	3 & Plot Modification
	& As a player, I must be able to buy and sell various modifications to my plots to increase productivity and/or style.
	& \begin{enumerate}[label=3.1.\arabic*.]
		\item The system must provide a number of possible modifications to plots
	\end{enumerate}
	& \begin{enumerate}[label=3.2.\arabic*.]
		\item The player must be able to view all modifications and choose one to install
		\item Installation must take less than a second
	\end{enumerate} \\ \hline

	4 & Multiplayer
	& As a player, I must be able to buy and sell various modifications to my plots to increase productivity and/or style.
	& \begin{enumerate}[label=4.1.\arabic*.]
		\item A player must be able to chose whether to play against another human or the computer
		\item At least two users must be able to play the game together
		\item The players will take turns in playing
	\end{enumerate}
	& \begin{enumerate}[label=4.2.\arabic*.]
		\item The simulated player should take no longer than 20 seconds to complete a round
	\end{enumerate} \\ \hline

	5 & Round Structure
	& As a player I must be able to play the game in a structured manner
	& \begin{enumerate}[label=5.1.\arabic*.]
		\item The game must be split into multiple rounds
		\item Each round should be made of 5 phases:
		\begin{enumerate}[label=\arabic*.]
			\item Purchase any unoccupied plots
			\item Purchase and customise roboticons
			\item Install roboticons on plots of land
			\item The colony produces resources
			\item The player can buy and sell resources
		\end{enumerate}
		\item Phases 2 \& 3 must be time limited. 
	\end{enumerate}
	& \begin{enumerate}[label=5.2.\arabic*.]
		\item It must be easy for the player to move between phases
		\item Changes between phases must take no longer than 5 seconds
	\end{enumerate} \\ \hline

	6 & Roboticons
	& As a player I must be able to purchase and customise my roboticons so they can produce more of certain amounts of resources
	& \begin{enumerate}[label=6.1.\arabic*.]
		\item The player must be able to purchase roboticons from the market
		\item The market must have ore to produce roboticons
		\item The user must be able to purchase modifications for the robiticon at the market
		\item The user must be able to install modifications on roboticons 
		\item The user must have the option to install a roboticon on a plot of land they own.
	\end{enumerate}
	& \begin{enumerate}[label=6.2.\arabic*.]
		\item At the start of the game, the market has 12 roboticons
		\item The user should be able to tell how a roboticon they are installing will affect the plot it is installed on
	\end{enumerate} \\ \hline

	7 & Resources
	& As a player I must be able to produce resources from my plots
	& \begin{enumerate}[label=7.1.\arabic*.]
		\item Roboticons are required to produce resources
		\item During phase 4 the user’s roboticons will generate resources across the freehold
		\item Food, energy and ore will be generated
		\item Different amount of resources will affect the rate of production
	\end{enumerate}
	& \begin{enumerate}[label=7.2.\arabic*.]
		\item The resource production should not take more than 5 seconds
		\item The resource production should happen automatically
	\end{enumerate} \\ \hline

	8 & Buying/selling resources
	& As a player, I must be able to buy and sell resources to other players through an auction, or to the market at a fixed price so that I can maximise my wealth and productivity.
	& \begin{enumerate}[label=8.1.\arabic*.]
		\item The system must provide an auction facility, where the other player and the market bid for resources
		\item The system must choose a market price based on resource abundance
		\item The player must be able to buy/sell resources from/to other players, or the market
	\end{enumerate}
	& \begin{enumerate}[label=8.2.\arabic*.]
		\item At the start of the game, the market must have 16 units of food and energy and 0 units of ore
		\item At the start of the game, the player must have a small amount of money
	\end{enumerate} \\ \hline

	9 & Gambling
	& As a player, I must be able to enter the bar and either win or lose money.
	& \begin{enumerate}[label=9.1.\arabic*.]
		\item The system must provide a minigame where the player can gamble with their money
	\end{enumerate}
	& \begin{enumerate}[label=9.2.\arabic*.]
		\item The minigame must give feedback on the money won or lost
	\end{enumerate} \\ \hline

	10 & Winning
	& As a player, I must be able to win or lose the game.
	& \begin{enumerate}[label=10.1.\arabic*.]
		\item The system must assign a value to each resource at the end of the game, from which a player's final wealth is calculated
		\item The game must end on the round in which the last plot of land has been allocated.
		\item The player with the highest final wealth must be declared the winner, and Vice-Chancellor of the colony
	\end{enumerate} & \\ \hline
	\end{longtable}
The main risks associated with these requirements are risks 3, 4, 5 \& 6 as certain requirements may not be able to be implemented due to available tools or staff ability, however risks 8 and 11 must also be considered as they may change the requirements themselves.
\end{landscape}
\restoregeometry

\bibliographystyle{ieeetr}
\bibliography{references.bib}